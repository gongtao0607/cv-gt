\cvbl{
\cventry{2012}{Autopilot System Hardware}{ZhongKeYinYi UAV Co.,Ltd}{}{}{
Selected sensors, desiged and installed autopilot system.\\
Ported Linux, wrote drivers, fixed bugs.\\
Made a PWM generator and trimmer on FPGA.}
}{
\cventry{2012}{自动驾驶仪硬件系统}{中科银翼无人机公司}{}{}{
选择传感器,设计和安装自驾仪控制系统.\\
移植Linux,编写驱动程序,修复bug.\\
使用FPGA制作可编程PWM生成,修正器.}
}
\cvbl{
\cventry{2012}{OpenRISC System on Chip(ORPSoC)}{}{}{}{
Ported ORPSoC to DE0-Nano development board.\\
Ported Linux, connected and drove on board devices.\\
Fixed bugs in ORPSoC and Linux and submited patches.}
}{
\cventry{2012}{OpenRISC片上系统(ORPSOC)}{}{}{}{
在DE0-Nano开发板上移植ORPSOC.\\
移植Linux,连接和驱动开发板的自带外设.\\
修复ORPSoC和Linux中的bug并提交patch.}
}
\cvbl{
\cventry{2012}{Server Maintaining}{ZhongKeYinYi UAV Co.,Ltd}{}{}{
Constructed the network infrastructure in company.\\
Installed and configured server, including web service, version control service, telephone switching and VoIP service, security surveillance system, netfilter firewall.}
}{
\cventry{2012}{网络服务器维护}{中科银翼无人机公司}{}{}{
安装搭建网络基础设施\\
搭建和配置服务器,包括网站服务,版本库服务,电话交换和VoIP服务,安防监控系统,Netfilter防火墙}
}
\cvbl{
\cventry{2011-2012}{Wind-Solor Hybrid Powered Battery Charger}{ZhongKeYinYi UAV Co.,Ltd}{}{}{
Designed circuit.\\
Designed and soldered sample PCB(Printed Circuit Board).\\
Wrote firmware, provided API for UI application.
}
}{
\cventry{2011-2012}{风光互补充电池充电器}{中科银翼无人机公司}{}{}{
设计硬件电路和电路板.\\
加工和焊接样板.\\
编写固件,向上层提供编程接口.}
}
\cvbl{
\cventry{2011}{Remote Sensing of Model Planes}{Fengru Cup}{}{}{
Designed hardware and firmware of sensor module using AVR microcontrollers.\\
Desinged hardware of reciving module, using V-USB open source firmware to send data via USB.\\
Wrote dll(dynamic link library) on PC to provide data for User-Interface.}
}{
\cventry{2011}{遥控飞机状态遥测}{冯如杯}{}{}{
设计传感器模块的硬件和固件,使用AVR单片机作为主控.\\
设计接收端的硬件,使用AVR的开源项目V-USB使之通过USB与PC通讯.\\
在PC上编写dll用于向用户界面提供数据.}
}
\cvbl{
\cventry{2011}{Research on powersaving of Intel SCC platform}{}{}{}{
Study the Intel SCC architecture and provided RCCE library.\\
Using RCCE API to adjust frenquency of cores running mpi program to recude energy consumpt.}
}{
\cventry{2011}{Intel SCC功耗管理研究}{系统结构研究所}{}{}{
学习Intel SCC结构和提供的RCCE库.\\
使用RCCE来动态调整mpi程序在每个核心上的能耗}
}
\cvbl{
\cventry{2011}{High-Performance-Computing's Nodes Managing}{}{}{}{
Installed and configured Ganglia and Nagios service on computing nodes.\\
Wrote php interface for submitting computing jobs and recieving results.}
}{
\cventry{2011}{高性能计算组节点管理}{系统结构研究所}{}{}{
在节点上安装和配置Ganglia和Nagios监视服务.\\
编写了php界面用于提交计算任务和取回运行结果.}
}
\cvbl{
\cventry{2010-2011}{Robot Vision Controller}{Beihang Robot Team for Robocon 2011}{}{}{
Ported Linux and OpenCV library to S3C2410 board.\\
Rewrote image capture code to support video format of using camera.}
}{
\cventry{2010-2011}{机器人视觉控制器}{Robocon 2011北航机器人队}{}{}{
在S3C2410开发板上移植Linux和OpenCV库.\\
重写了OpenCV中的图像捕获部分以支持使用的摄像头视频格式.}
}
