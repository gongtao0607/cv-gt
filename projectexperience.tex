\cvbl{
\cventry{2012}{Autopilot System Hardware}{\zkyyen{}}{}{}{
Selected sensors, desiged and installed autopilot system.\\
Ported Linux, wrote drivers, fixed bugs.\\
Design a PWM\footnotemark generator and trimmer on FPGA\footnotemark, wrote driver for Linux.}
\addtocounter{footnote}{-2}
\stepcounter{footnote}\footnotetext{PWM: Pulse-width modulation.}
\stepcounter{footnote}\footnotetext{FPGA: Field-Programmable Gate Array.}
}{
\cventry{2012}{自动驾驶仪硬件系统}{\zkyycn{}}{}{}{
选择传感器,设计和安装自驾仪控制系统.\\
移植Linux,编写驱动程序,修复bug.\\
在FPGA\footnotemark上设计制作可编程PWM\footnotemark生成,修正器,并编写Linux驱动程序.}
\addtocounter{footnote}{-2}
\stepcounter{footnote}\footnotetext{FPGA:现场可编程逻辑门阵列.}
\stepcounter{footnote}\footnotetext{PWM:脉冲宽度调制.}
}
\cvbl{
\cventry{2012}{OpenRISC System on Chip(ORPSoC)}{}{}{}{
Ported ORPSoC to DE0-Nano development board.\\
Ported Linux, drove on board devices.\\
Fixed bugs and submited patches.}
}{
\cventry{2012}{OpenRISC片上系统(ORPSoC)}{}{}{}{
在DE0-Nano开发板上移植ORPSoC.\\
移植Linux,驱动板载外设.\\
修复的bug并提交patch.}
}
\cvbl{
\cventry{2012}{Server Maintaining}{\zkyyen{}}{}{}{
Constructed the network infrastructure in company.\\
Installed and configured server, including web service, version control service, telephone switching and VoIP service, security surveillance system, netfilter firewall.}
}{
\cventry{2012}{网络服务器维护}{\zkyycn{}}{}{}{
安装搭建网络基础设施.\\
搭建和配置服务器,包括网站服务,版本库服务,电话交换和VoIP服务,安防监控系统,Netfilter防火墙.}
}
\cvbl{
\cventry{2011-2012}{Wind-Solor Hybrid Powered Battery Charger}{\zkyyen{}}{}{}{
Designed circuit.\\
Designed and soldered sample PCB(Printed Circuit Board).\\
Wrote firmware, provided API for UI application.
}
}{
\cventry{2011-2012}{风光互补充电池充电器}{\zkyycn{}}{}{}{
设计硬件电路和电路板.\\
加工和焊接样板.\\
编写固件,向上层提供编程接口.}
}
\cvbl{
\cventry{2011}{Remote Sensing of Model Planes}{Fengru Cup}{}{}{
Designed hardware and firmware of sensor module using AVR microcontrollers.\\
Desinged hardware of reciving module, using V-USB open source firmware to send data via USB.\\
Wrote dll(dynamic link library) on PC to provide data for User-Interface.}
}{
\cventry{2011}{遥控飞机状态遥测}{冯如杯}{}{}{
设计传感器模块的硬件和固件,使用AVR单片机作为主控.\\
设计接收端的硬件,使用AVR的开源项目V-USB使之通过USB与PC通讯.\\
在PC上编写dll用于向用户界面提供数据.}
}
\cvbl{
\cventry{2011}{Research on powersaving of Intel SCC\footnotemark platform}{\icaen{}}{}{}{
Study the Intel SCC architecture and RCCE library.\\
Using RCCE API to adjust frenquency of cores running mpi program to recude energy consumpt.}
\footnotetext{SCC: Intel's Single-chip Cloud Computer microprocessor.}
}{
\cventry{2011}{Intel SCC\footnotemark功耗管理研究}{\icacn{}}{}{}{
学习Intel SCC结构和提供的RCCE库.\\
使用RCCE来动态调整mpi程序在每个核心上的能耗.}
\footnotetext{SCC: 英特尔Single-chip Cloud Computer处理器.}
}
\cvbl{
\cventry{2011}{Research on directory organization of ccNUMA\footnotemark}{\icaen{}}{}{}{
Study source code of RSIM project.\\
Rewrote RSIM for simulate new directory organization.\\
Wrote scripts to run benchmarks on simulation automatically.}
\footnotetext{ccNUMA: cache coherent Non-Uniform Memory Access.}
}{
\cventry{2011}{ccNUMA\footnotemark目录结构研究}{\icacn{}}{}{}{
研究学习了RSIM 项目的代码.\\
修改RSIM 的代码用于仿真新的目录结构.\\
编写脚本用于自动化仿真测试.}
\footnotetext{ccNUMA: 一致性高速缓存非均匀存储访问模型.}
}
\cvbl{
\cventry{2011}{Research on interconnections architecture of many-core CMP\footnotemark}{\icaen{}}{}{}{
Study GEM5 project.\\
Used GEM5 to simulate designed interconnections architecture.}
\footnotetext{CMP: Chip MultiProcessor.}
}{
\cventry{2011}{众核处理器互连结构的研究}{\icacn{}}{}{}{
学习的GEM5项目.\\
使用GEM5测试设计的处理器互联结构.}
}
\cvbl{
\cventry{2011}{Research on Xen Virtual Machine Live Migration}{\icaen{}}{}{}{
Study Xen's developer API.\\
Wrote program to calculate dirty pages produced by a virtual machine.
}
}{
\cventry{2011}{Xen虚拟机热迁移研究}{\icacn{}}{}{}{
学习Xen开发函数.\\
编写程序统计虚拟机产生的内存脏页}
}
\cvbl{
\cventry{2011}{High-Performance-Computing's Nodes Managing}{\icaen{}}{}{}{
Installed and configured Ganglia and Nagios service on computing nodes.\\
Wrote php interface for submitting computing jobs and recieving results.}
}{
\cventry{2011}{高性能计算组节点管理}{\icacn{}}{}{}{
为计算节点配置Ganglia和Nagios监视服务.\\
编写php界面用于提交计算任务和取回运行结果.}
}
\cvbl{
\cventry{2010-2011}{Robot Vision Controller}{Beihang Robot Team for Robocon\footnotemark 2011}{}{}{
Ported Linux and OpenCV library to S3C2410 board.\\
Rewrote image capture code to support video format of using camera.}
\footnotetext{Robocon: The Asia-Pacific Robot Contest.}
}{
\cventry{2010-2011}{机器人视觉控制器}{Robocon\footnotemark 2011北航机器人队}{}{}{
在S3C2410开发板上移植Linux和OpenCV库.\\
重写OpenCV中的图像捕获部分以支持使用的摄像头视频格式.}
\footnotetext{Robocon: 亚太大学生机器人大赛.}
}
