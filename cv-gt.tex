%% start of file `cvbasic.tex'; shows exemplarily the use of koma-moderncvclassic
%% (c) Copyright 2010--2012 Salome Södergran (salome.soedergran@gmx.ch)
\documentclass[%
fontsize=11pt,%
a4paper,%
pagesize,%
headinclude,footinclude,%
headings=normal,%
]{scrartcl}
\usepackage[unicode=true,colorlinks=false,pdfborder={0 0 0}]{hyperref}
\newif\ifcven
\newif\ifcvcn
\newcommand{\cvbl}[2]{
\ifcven
#1
\fi
\ifcvcn
#2
\fi
}
\cvcntrue
\cventrue


%以下是字体设定
%\usepackage{fontspec}
%\usepackage[slantfont,boldfont,CJKnumber]{xeCJK}
%\setCJKmainfont{宋体}%指定一种字体
%\setCJKmainfont{Adobe Song Std L}
%\setCJKmainfont[ItalicFont=Adobe Kaiti Std R,BoldFont=Adobe Heiti Std R]{Adobe Song Std L}%指定加粗斜体
%\setCJKsansfont[ItalicFont=Adobe Kaiti Std R,BoldFont=Adobe Heiti Std R]{Adobe Song Std L}%指定加粗斜体
%\setCJKmonofont[ItalicFont=Adobe Kaiti Std R,BoldFont=Adobe Heiti Std R]{Adobe Song Std L}%指定加粗斜体
%\setCJKmainfont[ItalicFont=楷体,BoldFont=黑体]{宋体}
%\setmainfont{Sabon LT Std}
%\setmonofont[Scale=0.8]{Courier New}
%\setsansfont{Sabon LT Std}%指定加粗斜体

%下面两行用于自动换行
%\XeTeXlinebreaklocale "zh"
%\XeTeXlinebreakskip = 0pt plus 1pt

\usepackage[left=2cm,right=2cm,top=3cm,bottom=5cm]{geometry} % page layout
\usepackage{scrpage2}
%\usepackage[utf8]{inputenc}
%\usepackage[T1]{fontenc}
%\usepackage[american]{babel}

\tolerance=200 % white space
\clubpenalty = 1000 % orphans
\widowpenalty = 1000 % widows

% ===========================
%    VARIABLES
% ===========================
%    must be defined, BEFORE koma-moderncvclassic is loaded!

% address; not needed variables should be commented out
\renewcommand*{\title}{Lebenslauf}% für PDF
\providecommand*{\firstname}{}
\providecommand*{\familyname}{}
\cvbl{
\renewcommand*{\firstname}{TAO}
}{
\renewcommand*{\firstname}{龚}
}
\cvbl{
\renewcommand*{\familyname}{GONG}
}{
\renewcommand*{\familyname}{韬}
}
\newcommand*{\acadtitle}{\cvbl{Undergraduate}{本科}}
\newcommand*{\addressstreet}{\cvbl{No21 DORM, Beihang UNIV}{北京市,海淀区}}
\newcommand*{\addresscity}{\cvbl{Beijing, China}{北京航空航天大学21号宿舍楼}}
%\newcommand*{\address}[2]{\addressstreet{#1}\addresscity{#2}}
%\newcommand*{\mobile}{}
\newcommand*{\mobile}{+86 15210988463}
%\newcommand*{\phone}{}
%\newcommand*{\faxnr}{}
\newcommand*{\email}{gongtao0607@gmail.com}
%\newcommand*{\extrainfo}{}
%\renewcommand*{\quote}{}

% ===========================
%    LENGTHS
% ===========================
%    must be defined, BEFORE koma-moderncvclassic is loaded!

% left column width (default value: 2,79cm)
% uncomment the \newlength-command, and uncomment and adjust one of the \set...-commands to change the default value
\newlength\myhintscolumnwidth
% \settowidth\myhintscolumnwidth{Left column takes this text's width}
\setlength\myhintscolumnwidth{.1\textwidth}
% \setlength\myhintscolumnwidth{5cm}


% ===========================
%    KOMA-MODERNCVCLASSIC
% ===========================

\usepackage[myblue]{koma-moderncvclassic} % color theme as option; default = myblue; other predefined colors that may be used: red, green, blue, cyan, magenta, yellow, black, white, darkgray, gray, lightgray

% picture
\photo[noframe]{3cm}{picture.jpg}
% 'frame' gives a frame around the picture (=default), 'noframe' does not;
% '3cm' is the width the picture must be resized to;
% 'picture.jpg' is the name of the picture file


% ===========================
%    ADAPTIONS
% ===========================
%\renewcommand*{\familydefault}{\sfdefault}% default font sans-serif
%\renewcommand*{\addressfont}{\normalsize\sffamily\mdseries\slshape}% sans-serif font for address, too
%\renewcommand{\firstnamefont}{\fontsize{24}{26}\sffamily\mdseries\upshape} % name in smaller font
\newcommand*{\totalpagemark}{\usekomafont{pagenumber}\thepage/\pageref{lastpage}}% for page and pagetotal
 
% ===========================
%    HEAD- AND FOOTLINES
% ===========================
\pagestyle{scrheadings}
\clearscrheadfoot
\ifoot{CV~\firstname~\familyname}
\ofoot{\totalpagemark}
%\ihead{}
%\ohead{}

% ===========================
%    BIBLIOGRAPHY
% ===========================

\usepackage[backend=bibtex8,% or biber
style=authortitle,% 
sorting=ydnt,% sorted by year, descending
]{biblatex}
\bibliography{cv-gt.bib}
\defbibheading{bibliography}[\cvbl{Publications}{出版物}]{\section{#1}}

\usepackage{xpatch}

% Just for demonstration
\ExecuteBibliographyOptions{maxbibnames=99}%firstinits,maxbibnames=99}
\DeclareNameAlias{default}{last-first/first-last}

\newbool{bbx:emph}

\renewcommand*{\mkbibnamefirst}[1]{\ifbool{bbx:emph}{\emph{#1}}{#1}}
\renewcommand*{\mkbibnamelast}[1]{\ifbool{bbx:emph}{\emph{#1}}{#1}}
\renewcommand*{\mkbibnameprefix}[1]{\ifbool{bbx:emph}{\emph{#1}}{#1}}
\renewcommand*{\mkbibnameaffix}[1]{\ifbool{bbx:emph}{\emph{#1}}{#1}}

\newbibmacro*{name:emph}{%
  \ifboolexpr{ test {\ifcurrentname{author}} and
               test {\ifbibliography} and
%               test {\ifentrytype{inproceedings}} and
               (( test {\iffieldundef{usera}} and
                  test {\ifnumequal{\value{listcount}}{1}} ) or
                  test {\ifnumequal{\number\numexpr\thefield{usera}+0\relax}
                                   {\value{listcount}}} ) }
   {\global\booltrue{bbx:emph}}
   {\global\boolfalse{bbx:emph}}}

\xpretobibmacro{name:last}{\usebibmacro{name:emph}}{}{}
\xpretobibmacro{name:first-last}{\usebibmacro{name:emph}}{}{}
\xpretobibmacro{name:last-first}{\usebibmacro{name:emph}}{}{}    

% ==================================================================
%       DOCUMENT
% ==================================================================
\begin{document}

\maketitle

%\section{\cvbl{Personal Information}{个人信息}}
%\cvbl{
%\cvline{margintext}{linetext}
%}{
%\cvline{待填充}{blabla}
%}

\section{\cvbl{Education}{教育}}
\cvbl{
\cventry{2009-2013}{B.S(Computer Science)}{School of Advanced Engineering}{Beihang University}{}{}
}{
\cventry{2009-2013}{学士(计算机科学)}{高等工程学院}{北京航空航天大学}{}{}
}


\section{\cvbl{Job Training}{实习}}
\cvbl{
\cventry{2010-2011}{Assistant of Researcher}{Institution of Computer Architecture}{Beihang University}{}{}
}{
\cventry{2010-2011}{研究员助理}{系统结构研究所}{北京航空航天大学}{}{}
}
\cvbl{
\cventry{2012}{Leader of Electrical Institution}{ZhongKeYinYi UAV(Unmanned Aerial Vehicle) Co.,Ltd}{Beijing,China}{}{Designing PCB and Hardware, Porting Software, Writing Hardware Driver, Designing Software Framework, Server Maintenance}
}{
\cventry{2012}{电子部门主管}{中科银翼无人机公司}{北京}{}{负责电路设计,电路板设计加工,操作系统移植,驱动程序编写,设计软件框架。服务器维护}
}
%\cventry{years}{degree/jobtitle}{institution/employer}{localization}{optional: grade/...}{optional: comment/job description}



\section{\cvbl{Project Experience}{项目经历}}
\cvbl{
\cventry{2012}{Autopilot System Hardware}{\zkyyen{}}{}{}{
Selected sensors, desiged and installed autopilot system.\\
Ported Linux, wrote drivers, fixed bugs.\\
Design a PWM\footnotemark generator and trimmer on FPGA\footnotemark, wrote driver for Linux.}
\addtocounter{footnote}{-2}
\stepcounter{footnote}\footnotetext{PWM: Pulse-width modulation.}
\stepcounter{footnote}\footnotetext{FPGA: Field-Programmable Gate Array.}
}{
\cventry{2012}{自动驾驶仪硬件系统}{\zkyycn{}}{}{}{
选择传感器,设计和安装自驾仪控制系统.\\
移植Linux,编写驱动程序,修复bug.\\
在FPGA\footnotemark上设计制作可编程PWM\footnotemark生成,修正器,并编写Linux驱动程序.}
\addtocounter{footnote}{-2}
\stepcounter{footnote}\footnotetext{FPGA:现场可编程逻辑门阵列.}
\stepcounter{footnote}\footnotetext{PWM:脉冲宽度调制.}
}
\cvbl{
\cventry{2012}{OpenRISC System on Chip(ORPSoC)}{}{}{}{
Ported ORPSoC to DE0-Nano development board.\\
Ported Linux, drove on board devices.\\
Fixed bugs and submited patches.}
}{
\cventry{2012}{OpenRISC片上系统(ORPSoC)}{}{}{}{
在DE0-Nano开发板上移植ORPSoC.\\
移植Linux,驱动板载外设.\\
修复的bug并提交patch.}
}
\cvbl{
\cventry{2012}{Server Maintaining}{\zkyyen{}}{}{}{
Constructed the network infrastructure in company.\\
Installed and configured server, including web service, version control service, telephone switching and VoIP service, security surveillance system, netfilter firewall.}
}{
\cventry{2012}{网络服务器维护}{\zkyycn{}}{}{}{
安装搭建网络基础设施.\\
搭建和配置服务器,包括网站服务,版本库服务,电话交换和VoIP服务,安防监控系统,Netfilter防火墙.}
}
\cvbl{
\cventry{2011-2012}{Wind-Solor Hybrid Powered Battery Charger}{\zkyyen{}}{}{}{
Designed circuit.\\
Designed and soldered sample PCB(Printed Circuit Board).\\
Wrote firmware, provided API for UI application.
}
}{
\cventry{2011-2012}{风光互补充电池充电器}{\zkyycn{}}{}{}{
设计硬件电路和电路板.\\
加工和焊接样板.\\
编写固件,向上层提供编程接口.}
}
\cvbl{
\cventry{2011}{Remote Sensing of Model Planes}{Fengru Cup}{}{}{
Designed hardware and firmware of sensor module using AVR microcontrollers.\\
Desinged hardware of reciving module, using V-USB open source firmware to send data via USB.\\
Wrote dll(dynamic link library) on PC to provide data for User-Interface.}
}{
\cventry{2011}{遥控飞机状态遥测}{冯如杯}{}{}{
设计传感器模块的硬件和固件,使用AVR单片机作为主控.\\
设计接收端的硬件,使用AVR的开源项目V-USB使之通过USB与PC通讯.\\
在PC上编写dll用于向用户界面提供数据.}
}
\cvbl{
\cventry{2011}{Research on powersaving of Intel SCC\footnotemark platform}{\icaen{}}{}{}{
Study the Intel SCC architecture and RCCE library.\\
Using RCCE API to adjust frenquency of cores running mpi program to recude energy consumpt.}
\footnotetext{SCC: Intel's Single-chip Cloud Computer microprocessor.}
}{
\cventry{2011}{Intel SCC\footnotemark功耗管理研究}{\icacn{}}{}{}{
学习Intel SCC结构和提供的RCCE库.\\
使用RCCE来动态调整mpi程序在每个核心上的能耗.}
\footnotetext{SCC: 英特尔Single-chip Cloud Computer处理器.}
}
\cvbl{
\cventry{2011}{Research on directory organization of ccNUMA\footnotemark}{\icaen{}}{}{}{
Study source code of RSIM project.\\
Rewrote RSIM for simulate new directory organization.\\
Wrote scripts to run benchmarks on simulation automatically.}
\footnotetext{ccNUMA: cache coherent Non-Uniform Memory Access.}
}{
\cventry{2011}{ccNUMA\footnotemark目录结构研究}{\icacn{}}{}{}{
研究学习了RSIM 项目的代码.\\
修改RSIM 的代码用于仿真新的目录结构.\\
编写脚本用于自动化仿真测试.}
\footnotetext{ccNUMA: 一致性高速缓存非均匀存储访问模型.}
}
\cvbl{
\cventry{2011}{Research on interconnections architecture of many-core CMP\footnotemark}{\icaen{}}{}{}{
Study GEM5 project.\\
Used GEM5 to simulate designed interconnections architecture.}
\footnotetext{CMP: Chip MultiProcessor.}
}{
\cventry{2011}{众核处理器互连结构的研究\cite{xtorus}}{\icacn{}}{}{}{
学习的GEM5项目.\\
使用GEM5测试设计的处理器互联结构.}
}
\cvbl{
\cventry{2011}{Research on Xen Virtual Machine Live Migration\cite{xen1}}{\icaen{}}{}{}{
Research on Xen virtual machine live migration.\\
Study Xen's developer API.\\
Wrote program to calculate dirty pages produced by a virtual machine.
}
}{
\cventry{2011}{Xen虚拟机热迁移研究\cite{xen1}}{\icacn{}}{}{}{
研究Xen虚拟机热迁移技术.\\
学习Xen开发函数.\\
编写程序统计虚拟机产生的内存脏页}
}
\cvbl{
\cventry{2011}{High-Performance-Computing's Nodes Managing}{\icaen{}}{}{}{
Installed and configured Ganglia and Nagios service on computing nodes.\\
Wrote php interface for submitting computing jobs and recieving results.}
}{
\cventry{2011}{高性能计算组节点管理}{\icacn{}}{}{}{
为计算节点配置Ganglia和Nagios监视服务.\\
编写php界面用于提交计算任务和取回运行结果.}
}
\cvbl{
\cventry{2010-2011}{Robot Vision Controller}{Beihang Robot Team for Robocon\footnotemark 2011}{}{}{
Ported Linux and OpenCV library to S3C2410 board.\\
Rewrote image capture code to support video format of using camera.}
\footnotetext{Robocon: The Asia-Pacific Robot Contest.}
}{
\cventry{2010-2011}{机器人视觉控制器}{Robocon\footnotemark 2011北航机器人队}{}{}{
在S3C2410开发板上移植Linux和OpenCV库.\\
重写OpenCV中的图像捕获部分以支持使用的摄像头视频格式.}
\footnotetext{Robocon: 亚太大学生机器人大赛.}
}


\nocite{*}
\printbibliography

\section{\cvbl{Award}{获奖}}
\cvbl{
\cvline{2012}{4th Prize in Taiwan Innovative UAV Design Competition}
}{
\cvline{2012}{台湾无人飞机创意设计竞赛第四名}
}
\cvbl{
\cvline{2011}{2nd Prize in Fengru Cup}
}{
\cvline{2011}{冯如杯二等奖}
}
\cvbl{
\cvline{2010}{3rd Prize in Beihang Programming Contest}
}{
\cvline{2010}{北航程序设计大赛三等奖}
}
\cvbl{
\cvline{2009}{2rd Prize in Electronic Innovation Contes}
}{
\cvline{2009}{电子创新大赛二等奖}
}
\cvbl{
\cvline{2009}{1st Prize Special Freshman Scholarship}
}{
\cvline{2009}{新生优秀奖学金一等奖}
}


\section{\cvbl{Computer Knowledges}{计算机技能}}
\cvcomputer{\cvbl{Languages}{语言}}{C/C++}{\cvbl{Archetecture}{结构}}{8086,x86,x86\_64}
\cvcomputer{}{Pascal}					{}{Arm}
\cvcomputer{}{CUDA}					{}{OpenRISC}
\cvcomputer{}{Java}						{}{Mips}
\cvcomputer{}{Latex}					{}{AVR}
\cvcomputer{}{SQL}						{}{8051}
\cvcomputer{}{Verilog HDL}		{}{}
\cvcomputer{}{Bash}						{}{}
\cvcomputer{\cvbl{System}{系统}}{Windows}{}{}
\cvcomputer{}{Linux}{}{}
\cvcomputer{}{$\mu$CLinux}		{}{}
\cvcomputer{}{FreeRTOS}				{}{}
\cvcomputer{}{$\mu$C/OS}			{}{}
\cvcomputer{\cvbl{Software}{软件}}{Altium Designer}{}{}
\cvcomputer{}{Quartus}{}{}
\cvcomputer{}{Cadence}{}{}
\cvcomputer{}{Matlab}{}{}
\cvcomputer{}{Mathematica}{}{}
\cvcomputer{}{Multisim}{}{}
\cvcomputer{}{Proteus}{}{}


\section{\cvbl{Personal Interests}{个人兴趣}}
\cvbl{
\cvline{Algorithms}{Parallel computing, Machine vision.}
}{
\cvline{算法}{并行计算, 机器视觉}
}
\cvbl{
\cvline{Network}{Network construction, Server maintenance(Linux).}
}{
\cvline{网络}{网络搭建, 服务器维护(Linux)}
}
\cvbl{
\cvline{Software}{Linux, Embedded system, Software porting.}
}{
\cvline{软件}{Linux, 嵌入式系统, 软件移植.}
}
\cvbl{
\cvline{Hardware}{Circuit and Printed Ciruit Board designing, Soldering, FPGA designing, SoC on FPGA.}
}{
\cvline{硬件}{电路和电路板设计、焊接, FPGA设计, FPGA片上系统}
}
\cvbl{
\cvline{Autonomous System}{Robotics, Unmanned Aerial Vehicle}
}{
\cvline{自动系统}{机器人, 无人机}
}
\cvbl{
\cvline{Extra}{Latex, Piano and piano tunning, Brainwave researching, DIY, Outdoor sports}
}{
\cvline{其他}{Latex, 钢琴和调律, 脑电波研究, DIY ,户外运动}
}


\section{\cvbl{Social Works}{社会工作}}
\cvbl{
\cvline{}{Ubuntu's Chinese translation group}
}{
\cvline{}{Ubuntu 中文翻译组成员}
}
\cvbl{
\cvline{}{ArchWiki's Chinese translation}
}{
\cvline{}{ArchWiki(ArchLinux在线文档)中文翻译}
}
\cvbl{
\cvline{}{Contribute code to OpenRISC System on Chip community}
}{
\cvline{}{在OpenRISC片上系统(ORPSOC)社区贡献代码}
}
\cvbl{
\cvline{}{Contribute code to Linux community}
}{
\cvline{}{在Linux社区贡献代码}
}
\section{\cvbl{Languages}{语言}}
\cvlanguage{Chinese}{native}{}
\cvlanguage{English}{fair}{Good at academical reading}


\label{lastpage}% needed for computing pagetotal
\end{document}
